\documentclass[11pt]{article}

\usepackage{
    amssymb,
    amsmath,
    amsfonts,
    calc,
    eurosym,
    geometry,
    ulem,
    graphicx,
    caption,
    color,
    setspace,
    sectsty,
    comment,
    footmisc,
    caption,
    % natbib,
    pdflscape,
    subcaption,
    subfiles,
    titling,
    array,
    hyperref,
    booktabs,
    longtable,
    float,
    authblk,
    makecell,
    bbm,
    threeparttable}

\usepackage{pgfplots} 

\usepackage[
    backend=biber,
    style=nature,
    date=year,
    doi=true,
    isbn=false,
    url=false,
    eprint=false
]{biblatex}

\AtEveryBibitem{%
  \clearfield{note}%
}
\AtEveryCitekey{\clearlist{publisher}}
\AtEveryBibitem{\clearlist{publisher}}

\usepackage{pgf,tikz}
\usetikzlibrary{arrows, automata}
\usetikzlibrary{shapes.geometric,positioning}
\usetikzlibrary{positioning,calc}

\usepackage{siunitx}
\newcolumntype{d}{S[input-symbols = ()]}

\normalem

\renewcommand\Affilfont{\small\itshape}

\onehalfspacing
\newtheorem{theorem}{Theorem}
\newtheorem{corollary}{Corollary}
\newtheorem{proposition}{Proposition}
\newtheorem{definition}{Definition}

\newenvironment{proof}[1][Proof]{\noindent\textbf{#1.} }{\ \rule{0.5em}{0.5em}}

\newtheorem{hyp}{Hypothesis}
\newtheorem{subhyp}{Hypothesis}[hyp]
\renewcommand{\thesubhyp}{\thehyp\alph{subhyp}}

\newcommand{\red}[1]{{\color{red} #1}}
\newcommand{\blue}[1]{{\color{blue} #1}}

\newcolumntype{L}[1]{>{\raggedright\arraybackslash}m{#1}}
\newcolumntype{C}[1]{>{\centering\arraybackslash}m{#1}}
\newcolumntype{R}[1]{>{\raggedleft\arraybackslash}m{#1}}
\subsubsectionfont{\normalfont\itshape}

\usepackage{mathtools}

\usepackage{letltxmacro}
\LetLtxMacro\orgvdots\vdots
\LetLtxMacro\orgddots\ddots

\makeatletter
\DeclareRobustCommand\vdots{%
  \mathpalette\@vdots{}%
}
\newcommand*{\@vdots}[2]{%
  % #1: math style
  % #2: unused
  \sbox0{$#1\cdotp\cdotp\cdotp\m@th$}%
  \sbox2{$#1.\m@th$}%
  \vbox{%
    \dimen@=\wd0 %
    \advance\dimen@ -3\ht2 %
    \kern.5\dimen@
    % remove side bearings
    \dimen@=\wd2 %
    \advance\dimen@ -\ht2 %
    \dimen2=\wd0 %
    \advance\dimen2 -\dimen@
    \vbox to \dimen2{%
      \offinterlineskip
      \copy2 \vfill\copy2 \vfill\copy2 %
    }%
  }%
}
\DeclareRobustCommand\ddots{%
  \mathinner{%
    \mathpalette\@ddots{}%
    \mkern\thinmuskip
  }%
}
\newcommand*{\@ddots}[2]{%
  % #1: math style
  % #2: unused
  \sbox0{$#1\cdotp\cdotp\cdotp\m@th$}%
  \sbox2{$#1.\m@th$}%
  \vbox{%
    \dimen@=\wd0 %
    \advance\dimen@ -3\ht2 %
    \kern.5\dimen@
    % remove side bearings
    \dimen@=\wd2 %
    \advance\dimen@ -\ht2 %
    \dimen2=\wd0 %
    \advance\dimen2 -\dimen@
    \vbox to \dimen2{%
      \offinterlineskip
      \hbox{$#1\mathpunct{.}\m@th$}%
      \vfill
      \hbox{$#1\mathpunct{\kern\wd2}\mathpunct{.}\m@th$}%
      \vfill
      \hbox{$#1\mathpunct{\kern\wd2}\mathpunct{\kern\wd2}\mathpunct{.}\m@th$}%
    }%
  }%
}
\makeatother

\def\spacingset#1{\renewcommand{\baselinestretch}%
{#1}\small\normalsize} \spacingset{1.2}
\newcommand{\ode}[2]{\frac{d{#1}}{d{#2}}}
\DeclareMathOperator{\E}{E}
\DeclareMathOperator{\indep}{\perp\!\!\!\perp}
\def\expit{\mathrm{expit}}
\def\logit{\mathrm{logit}}
%\geometry{left=1.0in,right=1.0in,top=1.0in,bottom=1.0in}

% \addbibresource{tnd.bib}

\begin{document}
%TC:ignore
\begin{titlepage}
\title{Within-host viral kinetics after SARS-CoV-2 infection: a meta-analysis based on individual participant data}
\author[1]{Christopher Boyer\thanks{email: \href{mailto:cboyer@hsph.harvard.edu}{cboyer@hsph.harvard.edu}}}
\author[1,2]{Marc Lipsitch}
\affil[1]{Department of Epidemiology, Harvard T.H. Chan School of Public Health, Boston, MA.}
\affil[2]{Department of Immunology and Infectious Diseases, Harvard T.H. Chan School of Public Health, Boston, MA.}
\date{\today}
\maketitle
\newpage
\begin{abstract}
\textbf{Background:} 
Knowledge of the within-host kinetics of SARS-CoV-2 infection is essential for developing evidence-based policy to limit transmission. Ideally, such a policy would be informed by the probability that an infected person remains infectious given their history. However, estimating this probability is complicated because there are multiple markers of infectiousness and within-host longitudinal samples of marker trajectories, particularly those that adequately capture the proliferation phase, are rare.

\textbf{Methods:}
We synthesized data across 5 prospective, longitudinally sampled cohorts with multiple measures of viral shedding. Studies were eligible if they collected repeated measurements from PCR-confirmed SARS-CoV-2 infections and had a sampling design that could capture the proliferation phase of the virus. Studies were included if they had longitudinal data on one or more of the following: viral load quantified via RT-PCR, direct viral culture, or rapid antigen test results. Individual participant data were requested from all identified studies and harmonized. We used a Bayesian hierarchical model to estimate the time-varying probability of infectiousness, defined as a positive viral culture, given an individual's history. 

\textbf{Findings:}
We obtained 20,000 samples from 2,00X infections 

\textbf{Interpretation:}
Modeling the joint distribution allows us to infer the possible values of harder to measure proxies at any point during the infection based on all the available data. 
\noindent \\
\vspace{0in} \\
\noindent\textbf{Keywords:} 
\bigskip
\end{abstract}
\setcounter{page}{0}
\thispagestyle{empty}
\end{titlepage}
\pagebreak \newpage

\section{Introduction} \label{sec:introduction}


\section{Methods} \label{sec:methods}
\subsection{Search strategy and selection criteria} \label{sec:data}

\subsection{Data extraction and harmonization} \label{sec:data}

\subsection{The data} \label{sec:data}
We include data from publicly available longitudinal datasets from acute SARS-CoV-2 infections where participants were swabbed or tested repeatedly using multiple assays that measure viral shedding. We only include studies where there was a decent chance that the proliferation phase was adequately captured, such as from contact tracing or occupational health cohorts where surveillance testing was performed independent of symptoms. All studies were approved by institutional review boards and obtained written informed consent where applicable.

The Assessment of Transmission and Contagiousness of COVID-19 in Contacts (ATACCC) study was a longitudinal, prospective cohort study of community contacts of newly diagnosed, PCR-confirmed SARS-CoV-2 index cases in the United Kingdom spanning two separate enrollment periods: ATACCC1 enrolled contacts from Sept 13, 2020, to March 31, 2021, during the SARS-CoV-2 pre-alpha and alpha variant waves; and ATACCC2 enrolled contacts from May 24, 2021, to Oct 28, 2021, during the delta variant wave. Participants consisted of household and non-household exposed contacts aged 5 years or older who provided informed consent and agreed to complete symptom diary and self-swabbing of the upper respiratory tract for up to 20 days. We use data from 57 well-documented infections which included the growth phase. 

The UIUC cohort study was a longitudinal, prospective cohort at the University of Illinois at Urbana-Champaign. During the fall of 2020 and spring of 2021, all faculty, staff and students were required to undergo at least twice weekly quantitative PCR with reverse transcription (RT-qPCR) testing for SARS-CoV-2. Participants were enrolled if they reported a negative RT-qPCR test result in the past 7 days and were either (1) within 24 h of a positive RT-qPCR result or (2) within 5 days of exposure to someone with a confirmed positive RT-qPCR result and nasal and saliva samples were collected daily for up to 14 days. Participants also completed a daily online symptom survey. We use data from 60 well-documented infections which included the growth phase.

The NBA occupational cohort study was a longitudinal, prospective cohort study among players, staff, and affiliates of the National Basketball Association who were infected with SARS-CoV-2. Between March 11, 2020, and July 28, 2022, the NBA conducted regular surveillance for SARS-CoV-2 infection as part of an occupational health program. This included frequent viral testing (often daily during high community COVID-19 prevalence) using a variety of platforms, but primarily via nucleic acid amplification tests, as well as clinical assessment including case diagnosis and symptom tracking. To assess viral concentration, RT-qPCR tests were conducted when possible, using anterior nares and oropharyngeal swabs. Data on participant age and vaccination status were collected where possible. Viral lineages were assigned using whole-genome sequencing, when feasible. This resulted in a longitudinal dataset of 424,401 SARS-CoV-2 tests with clinical COVID-19 history and demographic information for 3021 individuals.

The SARS-CoV-2 Human Challenge Characterisation Study was a single-center, phase 1, open-label, human challenge trial. Healthy adults aged 18 to 30 years who were at low absolute risk of hospitalization or death and with no evidence of previous SARS-CoV-2 infection or vaccination were recruited between March and July 2021 and inoculated intranasally with 10 TCID50 of a wild-type SARS-CoV-2 virus. Participants were housed in a quarantine unit and prospectively followed for the length of infection and completed twice daily nasal and throat swabs as well as a symptom diary. We use data from 18 individuals in whom inoculation produced a well-tolerated infection. 

\begin{table}[p]
    \centering
    \small
    \begin{tabular}{lcccccccc}
     \toprule
     & \multicolumn{2}{c}{NBA} & \multicolumn{2}{c}{ATACCC} & \multicolumn{2}{c}{UIUC}  & \multicolumn{2}{c}{HCT} \\
      & N & \%& N & \% & N & \% & N & \% \\
     \midrule
     \textit{Individual characteristics} &  &  &  &  &  &  &  &  \\
     Age: [0,30) & 818 & 41.1 & 16 & 28.1 & 37 & 61.7 & 18 & 100.0 \\
     Age: [30,50) & 876 & 44.0 & 32 & 56.1 & 17 & 28.3 & 0 & 0.0 \\
     Age: [50,100) & 295 & 14.8 & 9 & 15.8 & 6 & 10.0 & 0 & 0.0 \\
     Recurrence: No & 193 & 90.3 & 57 & 100.0 & 60 & 100.0 & 18 & 100.0 \\
     Recurrence: Yes & 193 & 9.7 & 0 & 0.0 & 0 & 0.0 & 0 & 0.0 \\
     Variant: Pre-Alpha & 191 & 9.5 & 13 & 22.8 & 43 & 71.7 & 18 & 100.0 \\
     Variant: Alpha & 49 & 2.5 & 12 & 21.0 & 16 & 26.7 & 0 & 0.0 \\
     Variant: Delta & 191 & 9.5 & 25 & 43.8 & 0 & 0.0 & 0 & 0.0 \\
     Variant: Omicron & 1,400 & 70.4 & 0 & 0.0 & 0 & 0.0 & 0 & 0.0 \\
     Variant: BA.4/BA.5 & 71 & 3.6 & 0 & 0.0 & 0 & 0.0 & 0 & 0.0 \\
     Variant: Other & 278 & 14.0 & 0 & 0.0 & 1 & 1.6 & 0 & 0.0\\
     History: Unvaccinated & 94 & 4.7 & 24 & 53.4 & 60 & 100.0 & 18 & 100.0 \\
     History: Vaccinated boosted & 982 & 49.4 & 0 & 0.0 & 0 & 0.0 & 0 & 0.0 \\
     History: Vaccinated unboosted & 269 & 13.5 & 21 & 46.6 & 0 & 0.0 & 0 & 0.0\\
     History: Vaccinated unreported & 5 & 0.3 & 0 & 0.0 & 0 & 0.0 & 0 & 0.0\\
     History: Unreported & 579 & 29.1 & 0 & 0.0 & 0 & 0.0 & 0 & 0.0\\
     History: Boosted unreported primary & 20 & 1.2 & 0 &0.0 & 0 & 0.0 & 0 & 0.0\\
     \midrule
     \textit{Lab measurements} &  &  &  &  &  &  &  &  \\
        Ct values & 21,463 & & 638 & & 934 & & 684 \\
        Viral cultures & 0 & & 638 & & 934 & & 684 \\
        LFD values & 0 & & 638 & & 934 & & 684 \\
        Symptom diaries & 0 & & 638 & & 934 & & 684 \\
     \midrule
     Individuals & 1,989 &  & 57 & & 60 & & 18 \\
     \bottomrule
     \end{tabular}
    \label{tab:my_label}
\end{table}

\section{The model} \label{sec:model}
Our goal is to develop a generative model of within-host viral kinetics during an acute infection which can be fit to data and linked through an observational process to multiple measures of viral shedding. We define the following variables:
\begin{itemize}
    \item $V_t$ is infectious virus at time $t$,
    \item $R_t$ is viral RNA copies at time $t$,
    \item $Y_t$ is symptom onset (0/1) at time $t$,
    \item $Z_t$ is a vector of observed test or biomarker results at time $t$ including results from viral culture ($V^*_t$), RT-PCR ($R^*_t$), and lateral flow tests ($L^*_t$), e.g. $Z_t = (V^*_t, R^*_t, L^*_t)$.
    \item $X$ is a vector of individual covariates (e.g. sex, age, immune history),
    \item $S$ is a vector of test characteristics, such as swab type, gene target, etc.
\end{itemize}
where $t$ indexes time since inoculation and for notational convenience we suppress indexes for individuals $i$. We would like to characterize the joint distribution, 
$$
f(V_t, R_t, Y_t, Z_t | X, t, S; \theta)
$$
defining the within host trajectories of viral shedding and test or biomarker results over time, where $\theta$ is a vector of parameters (we use $\theta$ generically throughout to represent as yet to be defined model parameters). To specify the model, it is convenient to factorize the joint distribution into a product of conditional distributions. A natural factorization is:
\begin{align*}
    f(&V_t, R_t, Y_t, Z_t | X, S, t; \theta) = \\ & \qquad \underbrace{f(V_t | X, t; \theta)}_{\shortstack{infectious \\ virus}} \times \underbrace{f(R_t | V_t, X, t; \theta)}_{\shortstack{viral RNA}} \times \underbrace{f(Y_t | V_t, R_t, X, t; \theta)}_{\shortstack{natural history \\ of symptoms}} \times \underbrace{f(Z_t | V_t, R_t, Y_t, X, t, S; \theta)}_{\shortstack{observation \\ model}}
\end{align*}
where the first term describes the trajectory of infectious virus over time, the second term describes the total concentration of viral RNA over time conditional on the amount of infectious virus, the third term describes the onset of symptoms conditional on the amount of infectious virus and viral RNA, and the fourth term relates these to observed biomarker values and test results. We describe the models for each of these components in turn below.

\subsection{Infectious virus}
A common mechanistic model of   acute infection is the so-called target cell limited model. In its simplest form, the model includes compartments tracking the number of cells susceptible to infection ($T$), the number of productively infected cells ($I$), and the number of free virus particles ($V$) and is described by the following system of ordinary differential equations:
\begin{align*}
    \dfrac{dT}{dt}&= -b V T \\
    \dfrac{dI}{dt}&= b V T - d I \\
    \dfrac{dV}{dt}&= p I - c V
\end{align*}
Target cells become infected at rate $b$ through contact with free virus. Infected cells produce virus at rate $p$ and die at per capita rate $d$. Free virus is cleared at rate $c$. Extensions of this model include the addition of an eclipse phase, in which infected cells do not yet produce virus, and the inclusion of the immune response. An example numerical solution for the trajectory of viral particles over time is shown in Figure X. We note the following salient features: an initial exponential expansion of the infected cell population and number of free virus particles (proliferation), followed by a peak and subsequent exponential decline due to immune response and depletion of susceptible target cells (clearance).

Due to the complexity of the immune response and the difficulty of measuring the number of target or infected cells, it is difficult to specify a complete mechanistic model of acute infection. Therefore, we instead posit a semi-mechanistic model which retains many of the features of the target cell limited model, but is agnostic to the true structure of the response. Namely, we assume that the number of infectious viral particles follows a piece-wise exponential function, defined as
\begin{equation*}
    \log V_t = g(t; \theta),
\end{equation*}
where
\begin{equation*}
    g(t; \theta) = \begin{cases}
   \dfrac{\delta}{\omega_p} (t - (t_p - \omega_p)) & \text{if } t \leq t_p \\
    \delta - \dfrac{\delta}{\omega_r} (t - t_p) & \text{if } t > t_p,
\end{cases}
\end{equation*}
with parameters representing peak number of infectious virus ($\delta$), proliferation time ($\omega_p$), clearance time ($\omega_r$), and time of peak ($t_p$). As all of which can potentially vary with covariates $X$. The dotted line in Figure X shows an example trajectory of the number of infectious viral particles over time super-imposed on the numerical solution from the target cell limited model. A notable limitation is that the piece-wise exponential model tends to overestimate the peak value (in our experience by between X and Y\%). However, for our purposes it useful approximation which is easier to fit and directly parameterizes many of the quantities of interest, such as the time to peak and the time to clearance.

\subsection{Viral RNA copies}
Many tests, such as RT-PCR, detect or quantify viral RNA rather than infectious virus. That is, they do not distinguish between replication-competent (infectious) virus and residual (non-infectious) viral RNA which could be present due to production of nonviable viral particles during infection or may linger in a nonviable but not fully degraded state after the infection has been cleared by the immune system. In the target cell limited model, we could track the amount of viral RNA via additional compartments. For instance, we could add compartments tracking the number of  free floating RNA ($R$) via 
\begin{equation*}
    \dfrac{dR}{dt} = q p I - e R 
\end{equation*} 
where $q$ the amount of viral RNA per infectious virion and $e$ is the rate of degradation and clearance of viral RNA. However, we do not have direct measurements of the number of viral RNA copies, and the relationship between the number of infectious virus and the number of viral RNA copies is not well understood. Therefore, we again posit a semi-mechanistic model for the number of viral RNA copies. We assume that the number of RNA copies is a shifted version of the piece-wise exponential model for infectious virus
\begin{equation*}
    \log R_t = g(t; \theta^\prime),
\end{equation*}
where
\begin{equation*}
    g(t; \theta^\prime) = \begin{cases}
   \dfrac{\delta^\prime}{\omega^\prime_p} (t - (t^\prime_p - \omega^\prime_p)) & \text{if } t \leq t^\prime_p \\
    \delta^\prime - \dfrac{\delta^\prime}{\omega^\prime_r} (t - t^\prime_p) & \text{if } t > t^\prime_p,
\end{cases}
\end{equation*}
and
\begin{align*}
    \delta^\prime &= \delta^\prime_0 \exp(\tau_{\delta} \cdot \log \delta) \\
    \omega^\prime_{p} &= \omega^\prime_{p_0} \exp(\tau_{\omega_{p}} \cdot \log \omega_{p}) \\
    \omega^\prime_{r} &= \omega^\prime_{r_0} \exp(\tau_{\omega_{r}} \cdot \log \omega_{r}) \\
    t^\prime_{p} &= t^\prime_{p_0} + \tau_{t_p} \cdot t_{p} 
\end{align*}
where $\delta^\prime$, $\omega^\prime_{p}$, $\omega^\prime_{r}$, and $t^\prime_{p}$ are transformations of the corresponding parameters for the number of infectious virus. The transformations for $\delta^\prime$, $\omega^\prime_{p}$, and $\omega^\prime_{r}$ are on the log scale to ensure that they are strictly positive.

\subsection{Symptom onset}
The relationship between infectious virus and the onset of symptoms or symptom profile is not well understood. Therefore, we posit a statistical model for the onset of symptoms. Biologically, symptoms are a manifestation of the infection or the immune response, for which the number of infectious virus and the number of viral RNA copies are at least a proxy (and one we have information about). Thus, we assume that the discrete-time hazard for the onset of a particular symptom, $Y_{jt}$, is a function of $V_t$ and $R_t$ via the logistic model
\begin{equation*}
    Y_{j,t} \mid V_t, R_t, t, Y_{j,t-1} = 0 \sim \text{Bernoulli}(\text{logit}^{-1}(\eta_{0j} + \eta_{1j} \log V_t + \eta_{2j} \log R_t)).
\end{equation*}

\subsection{Observation models}
A number of viral culture assays seek to directly detect and quantify the number of infectious virus particles $V_t$. The simplest isolates the virus in cell culture by inoculating a sample onto a monolayer of susceptible cells and observing whether a cytopathic effect occurs. We model the probability of a positive viral culture result from this test as a function of the number of infectious virus particles via the logistic saturation model
\begin{equation*}
    V^{*}_{t,\text{culture}} \sim \text{Bernoulli}(\text{logit}^{-1}(\pi_0 + \pi_1 \log V_t)).
\end{equation*}
Alternatively, the number of infectious virus particles can be directly quantified by 50\% tissue culture infectious dose (TCID50), plaque forming units (PFU) assays, or focus-forming assays. For tests based on TCID50, we assume the number of days to a positive result is a function of the number of infectious virus particles in the innoculating sample and model the number of days to a positive result using the ordinal logistic model
\begin{equation*}
    V^{*}_{t,\text{TCID50}} \sim \text{Ordered-Logistic}(\log V_t, c).
\end{equation*}
where 
\begin{equation*}
    f(k; \log V_t, c) = 
    \begin{cases}
        1 - \text{logit}^{-1}(\log V_t - c_1) & \text{if } k = 1 \\
        \text{logit}^{-1}(\log V_t - c_{k-1}) - \text{logit}^{-1}(\log V_t - c_k) & \text{if } 1 < k < K \\
        \text{logit}^{-1}(\log V_t - c_{K-1}) & \text{if } k = K
    \end{cases}
\end{equation*}
i.e. $V_t$ is treated as a latent continuous variable which is censored at distinct thresholds, $c$, for each level of the ordinal outcome $k \in \{1, \ldots, K\}$. The plaque forming units (PFU) assay and focus-forming assays seek to directly characterize the amount of infectious virus in a sample by counting plaques or foci. We model the concentration of plaques or foci as a function of the true number of infectious virus particles via the log-normal model
\begin{align*}
    \log V^{*}_{t,\text{FFA}} &= \log V_t + \varepsilon_{\text{FFA}} \\
    \varepsilon_{\text{FFA}} &\sim \text{Normal}(0, \sigma_{\text{FFA}})_{lod} \\
    \log V^{*}_{t,\text{PFU}} &= \log V_t + \varepsilon_{\text{PFU}} \\
    \varepsilon_{\text{PFU}} &\sim \text{Normal}(0, \sigma_{\text{PFU}})
\end{align*}
where we assume errors are homoscedastic and censored at the level of detection for each assay. 

Much like simple viral isolation, qualitative RT-PCR tests provide information on the presence or absence of viral RNA in a sample, but not direct quantification. We model the probability of a positive RT-PCR test as a function of the number of viral RNA copies via the logistic model
\begin{equation*}
    R^{*}_{t,\text{PCR}} \sim \text{Bernoulli}(\text{logit}^{-1}(\pi_0 + \pi_1 \log R_t)).
\end{equation*}
By contrast, quantitative RT-PCR tests provide a cycle threshold (Ct) value, which is inversely related to the concentration of viral RNA in the clinical sample. Through calibration using an external standard with a defined number of RNA copies, the Ct value can be transformed into an estimate of the number of viral RNA copies via a characteristic curve. We model the number of viral RNA copies as a function of the Ct value via the log-normal model
\begin{align*}
    \log R^{*}_{t,\text{qPCR}} &= \log R_t + \varepsilon_{\text{qPCR}} \\
    \varepsilon_{\text{qPCR}} &\sim \text{Normal}(0, \sigma_{\text{qPCR}})_{lod}
\end{align*}
where, as with the PFU and FFA assays previously, we assume errors are homoscedastic and censored at the level of detection for each assay. Beyond simple measurement error of the quantitative result, a RT-PCR test can systematically fail due to sample quality or processing errors. We allow for false positive results by assuming that $R^{*}_{t,\text{qPCR}}$ are drawn from a mixture distribution
\begin{equation*}
    \log R^{*}_{t,\text{qPCR}} \sim \lambda \cdot \text{Normal}(\log R_t, \sigma_{\text{qPCR}}) + (1 - \lambda) \cdot \text{Exp}(1/\mu)
\end{equation*}
where $\lambda$ is the test specificity (assumed to be 0.99) and $\mu$ is the mean of the error distribution for false positive tests, which we set so that 90\% of distribution is within 1 Ct unit of the level of detection.

Lateral flow tests detect the presence of viral nucleocapsid proteins in a sample. The test is positive if the concentration of viral proteins exceeds a certain threshold. We assume the concentration of viral proteins is related to the number of infectious virus and the number of viral RNA copies, but do not directly model the true value. Instead, we model the probability of a positive antigen test as a function of the number of infectious virus particles and the number of viral RNA copies via the logistic model
\begin{equation*}
    L_t^* \sim \text{Bernoulli}(\text{logit}^{-1}(\gamma_0 + \gamma_1 \log V_t + \gamma_2 \log R_t)).
\end{equation*}

The gold standard for symptom data is a daily diary where participants record the presence or absence and severity across a range of symptoms, in which case we assume $Y^*_{jt} = Y_{jt}$. However, sometimes only information about the presence or absence of any symptoms (or a subset of symptoms) is available. In this case, we model the time to onset of any symptom via
\begin{equation*}
    Y^*_{t,\text{any}} \sim \text{Bernoulli}\left[1 - \prod_{j=1}^J \left\{1 - \text{logit}^{-1}(\eta_{0j} + \eta_{1j} \log V_t + \eta_{2j} \log R_t)\right\}\right].
\end{equation*}


\subsection{Covariate effects} 
It is well established that viral shedding varies with individual characteristics such as age, variant, and prior vaccination or infection history. Therefore, we allow the parameters of the piece-wise exponential model for the number of infectious virus particles and the number of viral RNA copies, as well as the parameters of the symptom onset and observation models, to vary with individual covariates $X$. In particular, we allow the parameters of $g(t; \theta)$ and $g(t; \theta^\prime)$ to vary with $X$ via
\begin{align*}
    \delta(X) &= \delta_0 \exp(\beta_\delta X) \\
    \omega_{p}(X) &= \omega_{p_0} \exp(\beta_{\omega_p} X) \\
    \omega_{r}(X) &= \omega_{r_0} \exp(\beta_{\omega_r} X)
\end{align*}
and 
\begin{align*}
    \delta^\prime(X) &= \delta^\prime_0 \exp\{\tau_{\delta} \cdot \delta(X) + \beta^\prime_\delta X\} \\
    \omega^\prime_{p_0}(X) &= \omega^\prime_{p} \exp\{\tau_{\omega_{p}} \cdot \omega_{p}(X) + \beta^\prime_{\omega_p} X\} \\
    \omega^\prime_{r_0}(X) &= \omega^\prime_{r} \exp\{\tau_{\omega_{r}} \cdot \omega_{r}(X) + \beta^\prime_{\omega_r} X\}
\end{align*}
as well as symptom onset via
\begin{equation*}
    Y_t \mid V_t, R_t, X, t, Y_{t-1} = 0 \sim \text{Bernoulli}(\text{logit}^{-1}(\pi_0 + \pi_1 \log V_t + \pi_2 \log R_t + \beta_{Y} X)).
\end{equation*}

We include in $X$: age (categorized as 0 to 30 years old, 30 to 50 years old, or 50+); variant (categorized as Pre-Alpha, Alpha, Delta, Omicron, BA.4/BA.5, or Other); a binary indicator of previous infection; and vaccination history (categorized as Unvaccinated, Vaccinated boosted, Vaccinated unboosted, Vaccinated unreported, Unreported, or Boosted unreported primary). For categorical variables, we use indicator coding with the first category as the reference level, in which case the reference model is for an unvaccinated and immunologically naive 0 to 30 year old infected with Pre-Alpha wildtype or variant.

\subsection{Individual and setting-specific random effects}
There is often residual variation in viral trajectories at the individual level beyond that which can be explained by the covariates in $X$. This could be due to heterogeneity in innoculating dose, differences in immune function, or other complex interactions between host and pathogen characteristics. We model residual variation in infectious virus and viral RNA at the individual level through the inclusion of individual-specific random effects for peak height, proliferation duration, clearance duration, and timing of peak, i.e. 
\begin{align*}
    \delta_0 &= \Delta_0 \exp(\delta_{0,i}) & \delta^\prime_0 &= \Delta^\prime_0 \exp(\delta^\prime_{0,i}) \\
    \omega_{p_0} &= \Omega_{p_0} \exp( \omega_{p_0,i}) & \omega^\prime_{p_0} &= \Omega^\prime_{p_0} \exp( \omega^\prime_{p_0,i}) \\
    \omega_{r_0}  &= \Omega_{r_0} \exp(\omega_{r_0,i}) & \omega^\prime_{r_0}  &= \Omega^\prime_{r_0} \exp(\omega^\prime_{r_0,i})
\end{align*}
where 
\begin{align*}
\delta_{0,i} &\sim N(0, \sigma_{\delta_0}) & \delta^\prime_{0,i} &\sim N(0, \sigma_{\delta^\prime_0}) \\
\omega_{p_0,i} &\sim N(0, \sigma_{\omega_{p_0}}) & \omega^\prime_{p_0,i} &\sim N(0, \sigma_{\omega^\prime_{p_0}}) \\
\omega_{r_0,i} &\sim N(0, \sigma_{\omega_{r_0}}) & \omega^\prime_{r_0,i} &\sim N(0, \sigma_{\omega^\prime_{r_0}}).
\end{align*}

When synthesizing results across settings additional variability may be present due to differences in measurement, such as the type of test used, the swab type, who does the swabbing, or the gene target, or differences in characteristics of participants, the pathogen, or other outbreak dynamics. We model this setting-specific variation through the inclusion of setting-specific random effects for the observation model parameters. 

\subsection{Missing data} 
We limit our sample to individuals with complete covariate information; however, some individuals do not have complete data for all tests or biomarkers at each time point. We assume that the missingness mechanism is ignorable conditional on covariates and therefore include the missing value as a parameter that is estimated in the posterior. That is, we define an indicator of whether the value is observed, $R_t$, and partition $Z_t$ into vectors of the observed $Z^{obs}_t$, the missing values $Z^{miss}_t$. The joint distribution of the observed and missing values is then
\begin{align*}
    f&(Z_t, R_t | H_t; \theta, \phi) = \\
    & \quad \int f(Z^{obs}_t, Z^{miss}_t | H_t; \theta) f(R_t = 0 | Z^{obs}_t, Z^{miss}_t, H_t; \phi) dZ^{miss}_t
\end{align*}
where we define $H_t = (V_t, R_t, Y_t, X, t, S)$ for notational convenience. We assume an ignorable mechanism conditional on covariates such that
\begin{equation*}
    f(R_t = 0 | Z^{obs}_t, Z^{miss}_t, H_t; \phi) = f(R_t = 0 | Z^{obs}_t, H_t; \phi)
\end{equation*}
and therefore 
\begin{equation*}
    f(Z_t, R_t | H_t; \theta, \phi) = f(Z^{obs}_t, R_t = 0 | H_t; \theta, \phi) 
\end{equation*}
implying that to learn about parameters it is sufficient estimate among the observed data. Given $Z^{miss}_t$ and $Z^{obs}_t$ are exchangeable conditional on covariates we estimate $Z^{miss}_t$ by drawing values from the posterior predictive distribution and using them. 
\begin{equation*}
    Z^{miss}_t \sim f(Z_t | V_t, R_t, Y_t, X, t, S; \theta)
\end{equation*}

\subsection{Priors}


    \begin{figure}
        \centering
        \begin{tikzpicture}
            \begin{axis}[
              no markers, domain=-3.5:4.5, samples=100,
              axis lines*=left, xlabel=days from highest value, ylabel=$ $,
              height=6cm, width=10cm,
              xtick={-5,-4,-3,-2,-1,0,1,2,3,4,5}, ytick={0,1,2,3,4,5,6,7,8,9,10},
              ymin = 0, ymax = 10, xmin =-3.5, xmax = 5,
              enlargelimits=false, clip=false, axis on top,
              grid = none, name=onset,
              declare function={
                pefunc(\x)=(\x <= -0.5) * (10 / 3 * (\x + 3.5))  +
                 (\x > -0.5) * (10 - 10 / 5 * (\x + 0.5));
                pefunc2(\x)=(\x <= -1) * (4 / 1 * (\x + 2))  +
                 (\x > -1) * (4 - 4 / 2 * (\x + 1));
              }
              ]
              % \addplot [draw=none, fill=blue!20] {lnormal(1.75,0.33)}\closedcycle;
              % \addplot [very thick, blue!50!black] {lnormal(1.75,0.33)};
            %   \addplot [draw=none, fill=orange!20] {gammapdf(4, 1)}\closedcycle;
            %   \addplot [very thick, orange!50!black] {gammapdf(4, 1)};
           
              \node (a) at (axis cs: -0.5,0) { };
              \node (a0) at (axis cs: -1,0) { };
              \node (b) at (axis cs: -1,4) { };
              \node (c) at (axis cs: -0.5,10) { };
              \node (d) at (axis cs: -2,0) { };
              \node (e) at (axis cs: 1,0) { };
              \node (f) at (axis cs: -3.5,0) { };
              \node (g) at (axis cs: 4.5,0) { };
              \node (h) at (axis cs: 0,0) { };
              \node (i) at (axis cs: 0,11) { };
              \node (j) at (axis cs: -1,1.5) { };
              \node (k) at (axis cs: 0,1.5) { };
              \draw[<->, very thick](a0)--(b);
              % \draw[<->, very thick](a)--(c);
              \draw[<->, very thick](a0)--(d);
              \draw[<->, very thick](a0)--(e);
              % \draw[<->, very thick](a)--(f);
              % \draw[<->, very thick](a)--(g);
               \draw[-, dashed](h)--(i);
              \draw[<->, thick](j)--(k);
              \node at (axis cs: -0.5,0.5) {$\omega^*_r$};
              % \node at (axis cs: 2,0.5) {$\omega_r$};
              \node at (axis cs: -1.5,0.5) {$\omega^*_p$};
              \node at (axis cs: -0.5,2) {$t^*_p$};
              % \node at (axis cs: -2.5,0.5) {$\omega_p$};
              \node at (axis cs: -1.25,2) {$\delta^*$};
              % \node at (axis cs: -0.75,6) {$\delta$};
              \addplot [very thick, red] {pefunc(x)};
              \addplot [very thick, blue!50!black, domain =-2:1] {pefunc2(x)};
              \node[red] at (axis cs: 3,5) {$\log R_t$};
              \node[blue!50!black] at (axis cs: 0.5,2.75) {$\log V_t$};
            \end{axis}
        \end{tikzpicture}
        \label{fig:illustration2}
    \end{figure}
    
\section{Computation}
lxret.
\section{Model checking and inference}


\section{Results}
\section{Discussion}

% \printbibliography


\clearpage

\begin{appendix}

    \renewcommand{\thefigure}{A\arabic{figure}}
    \setcounter{figure}{0}
    
    \renewcommand{\thetable}{A\arabic{table}}
    \setcounter{table}{0}
    
    \renewcommand{\theequation}{A\arabic{equation}}
    \setcounter{equation}{0}

%    \appendixwithtoc
    \newpage
    \begin{table}[p]
        \centering
        \caption{Posterior estimates of peak value.}
        \begin{tabular}{lccc}
        \toprule
         & \multicolumn{2}{c}{Peak value} \\
        Characteristic & $\exp(\beta)$ & 95\% CrI\\
        \midrule
         Age: [30,50) & 1.01 & (0.99, 1.02)\\
         Age: [50,100) & 1.00 & (0.98, 1.02)\\
         Recurrence & 0.95 & (0.92, 0.97)\\
         Variant: Alpha & 1.02 & (0.98, 1.06)\\
         Variant: Delta & 1.17 & (1.13, 1.21)\\
         Variant: Omicron & 1.05 & (1.02, 1.08)\\
         Variant: BA.4/BA.5 & 1.15 & (1.10, 1.21)\\
         Variant: other & 0.98 & (0.95, 1.02)\\
         History: Vaccinated boosted & 0.84 & (0.81, 0.87)\\
         History: Vaccinated unboosted & 0.86 & (0.83, 0.89)\\
         History: Vaccinated unreported & 0.83 & (0.80, 0.87)\\
         History: Unreported & 0.86 & (0.82, 0.91)\\
         History: Boosted unreported primary & 0.88 & (0.85, 0.90)\\
         \midrule
         Reference value, log [RNA] per ml & 17.22 & (16.81, 17.65)\\
         \bottomrule
        \end{tabular}
    \end{table}
    \begin{table}[p]
        \caption{Posterior estimates of proliferation phase duration.}
        \centering
        \begin{tabular}{lcc}
         \toprule
         & \multicolumn{2}{c}{Proliferation duration } \\
         Characteristic & $\exp(\beta)$ & 95\% CrI\\
         \midrule
         Age: [30,50) & 0.97 & (0.91, 1.04)\\
         Age: [50,100) & 1.08 & (0.99, 1.19)\\
         Recurrence & 0.86 & (0.77, 0.95)\\
         Variant: Alpha & 0.79 & (0.68, 0.91)\\
         Variant: Delta & 0.66 & (0.57, 0.75)\\
         Variant: Omicron & 0.93 & (0.82, 1.04)\\
         Variant: BA.4/BA.5 & 0.87 & (0.68, 1.14)\\
         Variant: other & 1.11 & (0.97, 1.26)\\
         History: Vaccinated boosted & 1.44 & (1.27, 1.64)\\
         History: Vaccinated unboosted & 1.17 & (1.02, 1.35)\\
         History: Vaccinated unreported & 1.22 & (1.03, 1.45)\\
         History: Unreported & 1.14 & (0.94, 1.39)\\
         History: Boosted unreported primary & 1.32 & (1.17, 1.50)\\
         \midrule
         Reference value, days & 7.29 & (6.61, 8.04)\\
         Reference value (lod), days & 4.67 & (4.23, 5.15)\\
         \bottomrule
         \end{tabular}
   \end{table}

 \begin{table}[p]
    \centering
    \caption{Posterior estimates of clearance phase duration.}
    \begin{tabular}{lcc}
     \toprule
     & \multicolumn{2}{c}{Clearance duration} \\
     Characteristic & $\exp(\beta)$ & 95\% CrI\\
     \midrule
     Age: [30,50) & 1.04 & (1.01, 1.08)\\
     Age: [50,100) & 1.19 & (1.13, 1.26)\\
     Recurrence & 0.74 & (0.70, 0.79)\\
     Variant: Alpha & 0.95 & (0.87, 1.04)\\
     Variant: Delta & 0.92 & (0.85, 1.00)\\
     Variant: Omicron & 0.90 & (0.84, 0.96)\\
     Variant: BA.4/BA.5 & 0.87 & (0.78, 0.97)\\
     Variant: other & 1.02 & (0.94, 1.10)\\
     History: Vaccinated boosted & 0.86 & (0.80, 0.93)\\
     History: Vaccinated unboosted & 0.74 & (0.68, 0.81)\\
     History: Vaccinated unreported & 0.79 & (0.72, 0.87)\\
     History: Unreported & 0.98 & (0.88, 1.10)\\
     History: Boosted unreported primary & 0.87 & (0.81, 0.93)\\
     \midrule
     Reference value, days & 14.55 & (13.69, 15.44)\\
     Reference value (lod), days & 9.31 & (7.96, 8.71)\\
     \bottomrule
     \end{tabular}
\end{table}

\begin{table}[p]
    \centering
    \caption{Correlation between [RNA] and viral culture}
    \begin{tabular}{lcccc}
     \toprule
     Parameter & $\exp(\rho)$ & 95\% CrI & $\theta$ & 95\% CrI\\
     \midrule
     Peak ($\delta^*$) & 0.47 & (0.45, 0.50) &  & \\
     Proliferation ($\omega_p^*$) & 0.50 & (0.43, 0.59) &  & \\
     Clearance ($\omega_r^*$) & 0.38 & (0.35, 0.42) &  & \\
     \midrule
     Peak time ($t_p^*$) & 1.37 & (1.15, 1.64) & -0.05 & (-0.32, 0.24)\\
     \bottomrule
     \end{tabular}
    \label{tab:corr}
\end{table}

\begin{table}[p]
    \centering
    \caption{LFD positivity as a function of [RNA] and viral culture}
    \begin{tabular}{lcc}
     \toprule
     Predictor & $\exp(\gamma)$ & 95\% CrI\\
     \midrule
     log RNA copies & 1.60 & (1.41, 1.80)\\
     log PFU culturable virus & 1.27 & (1.12, 1.45)\\
     \midrule
     Intercept & -6.41 & (-7.80, -5.00)\\
     \bottomrule
     \end{tabular}
    \label{tab:lfd}
\end{table}
\end{appendix}


\end{document}